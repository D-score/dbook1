\documentclass[]{book}
\usepackage{lmodern}
\usepackage{amssymb,amsmath}
\usepackage{ifxetex,ifluatex}
\usepackage{fixltx2e} % provides \textsubscript
\ifnum 0\ifxetex 1\fi\ifluatex 1\fi=0 % if pdftex
  \usepackage[T1]{fontenc}
  \usepackage[utf8]{inputenc}
\else % if luatex or xelatex
  \ifxetex
    \usepackage{mathspec}
  \else
    \usepackage{fontspec}
  \fi
  \defaultfontfeatures{Ligatures=TeX,Scale=MatchLowercase}
\fi
% use upquote if available, for straight quotes in verbatim environments
\IfFileExists{upquote.sty}{\usepackage{upquote}}{}
% use microtype if available
\IfFileExists{microtype.sty}{%
\usepackage{microtype}
\UseMicrotypeSet[protrusion]{basicmath} % disable protrusion for tt fonts
}{}
\usepackage[margin=1in]{geometry}
\usepackage{hyperref}
\hypersetup{unicode=true,
            pdftitle={D-score for measuring development of children 0-4 years},
            pdfauthor={Stef van Buuren},
            pdfborder={0 0 0},
            breaklinks=true}
\urlstyle{same}  % don't use monospace font for urls
\usepackage{natbib}
\bibliographystyle{apalike}
\usepackage{longtable,booktabs}
\usepackage{graphicx,grffile}
\makeatletter
\def\maxwidth{\ifdim\Gin@nat@width>\linewidth\linewidth\else\Gin@nat@width\fi}
\def\maxheight{\ifdim\Gin@nat@height>\textheight\textheight\else\Gin@nat@height\fi}
\makeatother
% Scale images if necessary, so that they will not overflow the page
% margins by default, and it is still possible to overwrite the defaults
% using explicit options in \includegraphics[width, height, ...]{}
\setkeys{Gin}{width=\maxwidth,height=\maxheight,keepaspectratio}
\IfFileExists{parskip.sty}{%
\usepackage{parskip}
}{% else
\setlength{\parindent}{0pt}
\setlength{\parskip}{6pt plus 2pt minus 1pt}
}
\setlength{\emergencystretch}{3em}  % prevent overfull lines
\providecommand{\tightlist}{%
  \setlength{\itemsep}{0pt}\setlength{\parskip}{0pt}}
\setcounter{secnumdepth}{5}
% Redefines (sub)paragraphs to behave more like sections
\ifx\paragraph\undefined\else
\let\oldparagraph\paragraph
\renewcommand{\paragraph}[1]{\oldparagraph{#1}\mbox{}}
\fi
\ifx\subparagraph\undefined\else
\let\oldsubparagraph\subparagraph
\renewcommand{\subparagraph}[1]{\oldsubparagraph{#1}\mbox{}}
\fi

%%% Use protect on footnotes to avoid problems with footnotes in titles
\let\rmarkdownfootnote\footnote%
\def\footnote{\protect\rmarkdownfootnote}

%%% Change title format to be more compact
\usepackage{titling}

% Create subtitle command for use in maketitle
\newcommand{\subtitle}[1]{
  \posttitle{
    \begin{center}\large#1\end{center}
    }
}

\setlength{\droptitle}{-2em}
  \title{D-score for measuring development of children 0-4 years}
  \pretitle{\vspace{\droptitle}\centering\huge}
  \posttitle{\par}
  \author{Stef van Buuren}
  \preauthor{\centering\large\emph}
  \postauthor{\par}
  \predate{\centering\large\emph}
  \postdate{\par}
  \date{2018-04-04}

\usepackage{booktabs}

\begin{document}
\maketitle

{
\setcounter{tocdepth}{1}
\tableofcontents
}
\chapter*{Preface}\label{preface}
\addcontentsline{toc}{chapter}{Preface}

This is an introductory booklet on the measurement of child development
by means of the D-score. The D-score is a one-number summary that
quantifies generic neurocognitive development for children with ages 0-4
years.

This is the \emph{first} in a series of three booklets. The series
consists of the following titles:

\begin{enumerate}
\def\labelenumi{\arabic{enumi}.}
\tightlist
\item
  \href{https://stefvanbuuren.github.io/dbook1/}{D-score for measuring
  development of children 0-4 years}
\item
  \href{https://stefvanbuuren.github.io/dbook2/}{D-score for
  international comparisons}
\item
  D-score for creating better instruments
\end{enumerate}

The development of this series is kindly supported by the Bill \&
Melinda Gates Foundation.

\chapter{Introduction}\label{intro}

\section{First 1000 days}\label{first-1000-days}

\section{Relevance of child
development}\label{relevance-of-child-development}

\section{Limitations of stunting}\label{limitations-of-stunting}

\section{Measuring neurocognitive
development}\label{measuring-neurocognitive-development}

\chapter{Short history}\label{short-history}

\section{Growth and development}\label{growth-and-development}

\section{Gesell maturation theory, Piaget stages, Kohlberg
stages}\label{gesell-maturation-theory-piaget-stages-kohlberg-stages}

\section{One number for development}\label{one-number-for-development}

\section{Current situation: Bayley, Griffiths, IQ,
domains}\label{current-situation-bayley-griffiths-iq-domains}

\chapter{Comparisons}\label{comparisons}

\section{Types of comparisons needed}\label{types-of-comparisons-needed}

\section{Problems of age-based
measurement}\label{problems-of-age-based-measurement}

\section{What is a latent variable}\label{what-is-a-latent-variable}

\section{Item response functions}\label{item-response-functions}

\section{Person response functions}\label{person-response-functions}

\section{Family of IRT models}\label{family-of-irt-models}

\chapter{Rasch model}\label{rasch}

\section{Rasch model}\label{rasch-model}

\section{Perfect symmetry}\label{perfect-symmetry}

\section{Parameter separation}\label{parameter-separation}

\section{The model as ideal}\label{the-model-as-ideal}

\chapter{Items}\label{items}

\section{SMOCC data: design}\label{smocc-data-design}

\section{Empirical and fitted item response
curves}\label{empirical-and-fitted-item-response-curves}

\section{Item fit}\label{item-fit}

\section{Item information at a given
ability}\label{item-information-at-a-given-ability}

\section{Item information at a given
age}\label{item-information-at-a-given-age}

\chapter{Persons}\label{persons}

\section{Empirical and fitted person response
curves}\label{empirical-and-fitted-person-response-curves}

\section{Person fit}\label{person-fit}

\section{Ability estimation}\label{ability-estimation}

\section{Measurement precision}\label{measurement-precision}

\section{Distribution of ability against
age}\label{distribution-of-ability-against-age}

\chapter{Validity}\label{validity}

\section{Role of validity}\label{role-of-validity}

\section{Discriminatory validity}\label{discriminatory-validity}

\section{Concurrent validity}\label{concurrent-validity}

\section{Predictive validity}\label{predictive-validity}

\chapter{Outcome}\label{outcome}

\section{Application I: D-score as neurocognitive outcome at 1000
days}\label{application-i-d-score-as-neurocognitive-outcome-at-1000-days}

\section{D-score of reference children at 2
years}\label{d-score-of-reference-children-at-2-years}

\section{D-score of pre-terms at 2
years}\label{d-score-of-pre-terms-at-2-years}

\section{D-score of children in LMIC at 2
years}\label{d-score-of-children-in-lmic-at-2-years}

\section{Comparison}\label{comparison}

\chapter{Delay}\label{delay}

\section{Application II: D-score to identify delayed
development}\label{application-ii-d-score-to-identify-delayed-development}

\section{Longitudinal D-score patterns in different
populations}\label{longitudinal-d-score-patterns-in-different-populations}

\section{Issues in defining developmental
delay}\label{issues-in-defining-developmental-delay}

\section{Specificity in reference, pre-term and LMIC
populations}\label{specificity-in-reference-pre-term-and-lmic-populations}

\section{Practical implications}\label{practical-implications}

\chapter{Consequences}\label{consequences}

\section{Application III: Long-term health consequences of delay in
pre-terms}\label{application-iii-long-term-health-consequences-of-delay-in-pre-terms}

\section{Relevance of long-term health
outcomes}\label{relevance-of-long-term-health-outcomes}

\section{Predictive power of D-score}\label{predictive-power-of-d-score}

\section{Practical implications}\label{practical-implications-1}

\section{Opportunities and impact of early
intervention}\label{opportunities-and-impact-of-early-intervention}

\chapter{Discussion}\label{discussion}

\section{Usefulness of D-score for monitoring child
health}\label{usefulness-of-d-score-for-monitoring-child-health}

\section{Opportunities for early
intervention}\label{opportunities-for-early-intervention}

\section{D-score for international
settings}\label{d-score-for-international-settings}

\section{D-score from existing
instruments}\label{d-score-from-existing-instruments}

\section{Creating new instruments for
D-score}\label{creating-new-instruments-for-d-score}

\bibliography{book.bib}


\end{document}
